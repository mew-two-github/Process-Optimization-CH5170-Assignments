\documentclass{homework}

\title{CH5170 Assignment-1}
\author{S. VISHAL (CH18B020)}

\begin{document}

\maketitle
\section{Problem Details}
\begin{figure}[ht]
\centering
\includegraphics[width=0.75\textwidth]{question.jpg}
\caption{Pressure Values obtained from Geogebra}
\end{figure}
Other data as obtained from the Problem Sheet are:
\begin{enumerate}
\item $H_A$ = 100 m
\item Lengths of the links $L_{1}$ = 300 m, $L_{1}$ = 300 m, $L_{1}$ = 300 m. \item Flow rates in the links are $Q_1$ = 9 m$\m^3$/min, $Q_2$ = 3 m$\m^3$/min and $Q_3$ = 2 m$\m^3$/min respectively.
\end{enumerate}
Finally, cost of a pipe can be estimated using:
\begin{equation}\label{pipe_cost}
c = 1.2654LD^{1.327}
\end{equation}
where L is the length of the pipe (in m), and D is the diameter of the pipe (in mm)
\exercise*[1: Head Losses]
We make use of the following expression for head loss:
\begin{equation}\label{head_loss}
\Delta H = 4.457*10^8*\frac{LQ^{1.85}}{D^{4.87}}
\end{equation}
We can first write the equation at node B as:
\begin{equation}\label{node_B_Pressure}
H_A - H_B = 4.457*10^8*\frac{L_1Q_1^{1.85}}{D_1^{4.87}}
\Rightarrow{H_B = H_A - 4.457*10^8*\frac{L_1Q_1^{1.85}}{D_1^{4.87}}}
\end{equation}
For node C,
\begin{equation}\label{node_C_Pressure}
H_B - H_C = 4.457*10^8*\frac{L_2Q_2^{1.85}}{D_2^{4.87}}
\end{equation}
Substituting from \ref{node_B_Pressure} we get $H_C$ as:
\begin{equation}\label{node_C_Psub}
H_C = H_A - 4.457*10^8*\frac{L_1Q_1^{1.85}}{D_1^{4.87}}-4.457*10^8*\frac{L_2Q_2^{1.85}}{D_2^{4.87}}
\end{equation}
Proceeding similarly for node D:
\begin{equation}\label{node_D_Pressure}
H_B - H_D = 4.457*10^8*\frac{L_3Q_3^{1.85}}{D_3^{4.87}}
\end{equation}
\begin{equation}\label{node_D_Psub}
\Rightarrow H_D = H_A - 4.457*10^8*\frac{L_1Q_1^{1.85}}{D_1^{4.87}}-4.457*10^8*\frac{L_3Q_3^{1.85}}{D_3^{4.87}}
\end{equation}
Thus we have 3 equations (\ref{node_B_Pressure}, \ref{node_C_Psub} and \ref{node_D_Psub}) which relate the Head losses to some known parameters and the pipe diameters($D_1$, $D_2$ and $D_3$).
\exercise*[2: Total Cost]
We can simply use cost equation (eqn \ref{head_loss}) for all the three pipes:

Cost of link 1 (diameter $D_1$ and length $L_1$) is $c_1L_1 = 1.2654L_1D_1^{1.327}$

Cost of link 2 (diameter $D_2$ and length $L_2$) is $c_2L_2 = 1.2654L_2D_2^{1.327}$

Cost of link 3 (diameter $D_3$ and length $L_3$) is $c_3L_3 = 1.2654L_3D_3^{1.327}$

Adding them up we get,
\begin{equation}\label{Total_cost}
Cost = 1.2654L_1D_1^{1.327} + 1.2654L_2D_2^{1.327} + 1.2654L_3D_3^{1.327}
\end{equation}

\exercise[3: Cost in terms of Pressures]
Rearranging the Head Loss equation, and obtaining the diameter, we have from eqn \ref{head_loss}
\[D = (4.457*10^8*\frac{LQ^{1.85}}{\Delta{H}})^{\frac{1}{4.87}}\]
Proceeding same way for all 3 pipes we obtain:
\begin{equation}\label{D1}
D_1 = (4.457*10^8*\frac{L_1Q_1^{1.85}}{H_A-H_B})^{\frac{1}{4.87}}
\end{equation}
\begin{equation}\label{D2}
D_2 = (4.457*10^8*\frac{L_2Q_2^{1.85}}{H_B-H_C})^{\frac{1}{4.87}}
\end{equation}
\begin{equation}\label{D3}
D_3 = (4.457*10^8*\frac{L_3Q_3^{1.85}}{H_B-H_D})^{\frac{1}{4.87}}
\end{equation}
Substituting diameter values from equations \ref{D1},\ref{D2},\ref{D3} into the cost equation obtained previously (eqn \ref{Total_cost}), we obtain:

\begin{equation}\label{Cost_Head}
\begin{split}
Total cost = 1.2654*L_1*4.457*10^{8}*(\frac{L_1Q_1^{1.85}}{H_A - H_B})^{\frac{1.327}{4.87}} 
    & + 1.2654*L_2*4.457 * 10^{8}*(\frac{L_2Q_2^{1.85}}{H_B - H_C})^{\frac{1.327}{4.87}}
    & + 1.2654*L_3*4.457 * 10^{8}*(\frac{L_3Q_3^{1.85}}{H_B-H_D})^{\frac{1.327}{4.87}}
\end{split}
\end{equation}

\exercise*[4: Optimization Formulation]
Our objective is to minimize the cost function which we derived above.
The objective function:
\begin{equation}\label{Objective}
\begin{split}
min_{H_B,H_C,H_D} Total cost = 1.2654*L_1*4.457*10^{8}*(\frac{L_1Q_1^{1.85}}{H_A - H_B})^{\frac{1.327}{4.87}} 
    & + 1.2654*L_2*4.457 * 10^{8}*(\frac{L_2Q_2^{1.85}}{H_B - H_C})^{\frac{1.327}{4.87}}
    & + 1.2654*L_3*4.457 * 10^{8}*(\frac{L_3Q_3^{1.85}}{H_B-H_D})^{\frac{1.327}{4.87}}
\end{split}
\end{equation}
such that (Inequality constraints)
\begin{equation}\label{Ineqc}
\begin{split}
H_B \geq 79.5m\\
H_C \geq 89m   \\
H_D \geq 81.5m
\end{split}
\end{equation}
And the diameters so obtained should be positive. (bound constraint; which upon substituting in the diameter in terms of head-loss equation \ref{node_B_Pressure}, \ref{node_C_Pressure} and \ref{node_D_Pressure} implies $H_A > H_B, H_B > H_C$ and $H_B > H_D$)

\exercise*[5: Optimum Pressures]
From the inequality and equality constraints we can infer 2 things:$\\\\$
1. \textbf{Inequality constraint-1 is redundant:} $\\$
For water to flow to tanks C and D from B, we know that $H_B > H_C$ and $H_B > H_D.\\$
But $H_C > 89 and H_D > 81.5$ which are higher than the limit 79.5 m imposed on $H_A$ as we can see from eqn \ref{Ineqc}. Therefore, Inequality constraint-1 is redundant. Imposing $H_C > 89m$ itself ensures $H_B > 89m\\$
2. \textbf{Optimal values of $H_C$ and $H_D$:} $\\$
Consider the equation \ref{node_C_Pressure}
\[H_B-H_C = 4.457*10^8*\frac{L_2Q_2^{1.85}}{D_2^{4.87}}\]
Let $H_C$ be some 100 m. If I keep decreasing $D_2$, the pressure head also keeps falling. Lower the diameter, lower the cost. So we keep lowering the diameter until we hit the minimum value for $H_C$. At this point, we can't reduce pipe size and hence the pipe cost for link II can't be reduced further.$\\$
\textbf{Therefore, $H_C$ = 89m is the optimal Pressure at node C.}$\\$
\textbf{By a similar argument, $H_D$ = 84m is the optimal Pressure at node D.} $\\$
As a result of the above inferences, we can remove all the 3 inequality constraints (automatically satisfied). We can then substitute values of $H_C$ and $H_D$ in equation \ref{Cost_Head} as 89m and 81.5m. \\
Now the objective is 
\begin{equation}\label{Cost_Head}
\begin{split}
Total cost = 1.2654*L_1*4.457*10^{8}*(\frac{L_1Q_1^{1.85}}{H_A - H_B})^{\frac{1.327}{4.87}} 
    & + 1.2654*L_2*4.457 * 10^{8}*(\frac{L_2Q_2^{1.85}}{H_B - H_C})^{\frac{1.327}{4.87}}
    & + 1.2654*L_3*4.457 * 10^{8}*(\frac{L_3Q_3^{1.85}}{H_B-H_D})^{\frac{1.327}{4.87}}
\end{split}
\end{equation}
Where $H_B$ is the only unknown and there are no constraints.

\exercise*[6: Unconstrained univariate optimisation]
The above optimisation problem was solved in MATLAB and the solution was found to be $H_B = 95.204m$ \\ And of course, as mentioned earlier, $H_C = 89m$ and $H_D = 81.5m$. \\
The corresponding pipe diameters obtained by substituting Pressure Heards in the equations from part one. \\
Diameter values:
\[D_1 = 321.5947 mm\]
\[D_2 = 223.1900 mm\]
\[D_3 = 155.3124 mm\]
Cost = Rs. $2.1096 * 10^6$\\
\textbf{MATLAB CODE:}
    \begin{verbatim}
clear;
% minimise costs
sol = fmincon(@cost,90,-1,80);
cost_sol = cost(sol);
L1 = 300; L2 = 500; L3 = 400;
Ho = 100; beta = 89; gamma = 81.5;
Q1  = 9; Q2 = 3; Q3 = 2;
D1 = (4.457*10^8)^(1/4.87)*(L1*Q1^1.85/(Ho-sol))^(1/4.87);
D2 = (4.457*10^8)^(1/4.87)*(L2*Q2^1.85/(sol-beta))^(1/4.87);
D3 = (4.457*10^8)^(1/4.87)*(L3*Q3^1.85/(sol-gamma))^(1/4.87);
function f = cost(HA)
    L1 = 300; L2 = 500; L3 = 450;
    Ho = 100; beta = 89; gamma = 81.5;
    Q1  = 9; Q2 = 3; Q3 = 2;
    f = 1.2654*L1*(4.457*10^8)^(1.327/4.87)*((L1*Q1^1.85/(Ho-HA)))^(1.327/4.87);
    f = f + 1.2654*L2*(4.457*10^8)^(1.327/4.87)*((L2*Q2^1.85/(HA-beta)))^(1.327/4.87);
    f = f + 1.2654*L3*(4.457*10^8)^(1.327/4.87)*((L3*Q3^1.85/(HA-gamma)))^(1.327/4.87);
end
\end{verbatim}

\exercise*[7: Optimisation for the case of Discrete Diameters]
Pipes of only a certain set of diameters are available in the market.
Since we can use two pipes per link, I am going to use two closest diameter
pipes in series as mentioned in the \emph{'problem set 1.pdf'}. This helps us get as close to a single pipe cost as possible.(because larger pipe is costlier than the optimal, and smaller pipe is cheaper than optimal[but can't maintain constraint]) Also, we need to ensure that the pressure constraints are satisfied. \\\\
Let the first link have two pipes of lengths $l_{11}$ and $l_{12}$, and the corresponding diameters be $d_{11}$ and $d_{12}$. Similarly, we define $l_{21}$, $l_{22}$, $l_{31}$, $l_{32}$, $d_{21}$, $d_{22}$, $d_{31}$, $d_{32}$. \\\\
For link 1 the diameters are:\( d_{11} = 300 mm\) \(d_{12} = 350 mm\). \\
For link 2 the diameters are:\( d_{21} = 200 mm\) \(d_{22} = 250 mm\). \\
For link 3 the diameters are:\( d_{31} = 150 mm\) \(d_{32} = 200 mm\). \\\\
Since, we are now using two pipes per link the total cost equation (eq. \ref{Total_cost}) is suitably modified as:
\begin{equation}\label{NewCost}
\begin{split}
Cost = 1.2654l_{11}d_{11}^{1.327} + 1.2654l_{12}d_{12}^{1.327} \\
+ 1.2654l_{21}d_{21}^{1.327} + 1.2654l_{22}d_{22}^{1.327} \\
+ 1.2654l_{31}d_{31}^{1.327} + 1.2654l_{32}d_{32}^{1.327} \\
\end{split}
\end{equation}
Further we have the following equality constraints:
\begin{equation}\label{NewEC}
\begin{split}
 L_1 = l_{11} + l_{12} \Rightarrow l_{12} = L_1 - l_{11} \\
 L_2 = l_{21} + l_{22} \Rightarrow l_{22} = L_2 - l_{21} \\
 L_3 = l_{31} + l_{32} \Rightarrow l_{32} = L_3 - l_{31}  \\
\end{split}
\end{equation}
So we can simplify the cost (eq. \ref{NewCost}) to
\begin{equation}\label{NewEC}
\begin{split}
Cost = 1.2654l_{11}d_{11}^{1.327} + 1.2654(L_1 - l_{11})d_{12}^{1.327} \\
+ 1.2654l_{21}d_{21}^{1.327} + 1.2654(L_2 - l_{21})d_{22}^{1.327} \\
+ 1.2654l_{31}d_{31}^{1.327} + 1.2654(L_3 - l_{31})d_{32}^{1.327} \\
\end{split}
\end{equation}
We can write equations relating pressure and diameter similar to those in \ref{node_B_Pressure}, \ref{node_C_Psub}, \ref{node_D_Psub}. They come out to be:
\begin{equation}
H_B = H_A - 4.457*10^8*(\frac{l_{11}Q_1^{1.85}}{d_{11}^{4.87}} + \frac{l_{12}Q_1^{1.85}}{d_{12}^{4.87}})
\end{equation}
\begin{equation}
H_C = H_A - 4.457*10^8*(\frac{l_{11}Q_1^{1.85}}{d_{11}^{4.87}} + \frac{l_{12}Q_1^{1.85}}{d_{12}^{4.87}} + \frac{l_{21}Q_2^{1.85}}{d_{21}^{4.87}} + \frac{l_{22}Q_2^{1.85}}{d_{22}^{4.87}})
\end{equation}
\begin{equation}
H_D = H_A - 4.457*10^8*(\frac{l_{11}Q_1^{1.85}}{d_{11}^{4.87}} + \frac{l_{12}Q_1^{1.85}}{d_{12}^{4.87}} + \frac{l_{31}Q_2^{1.85}}{d_{31}^{4.87}} + \frac{l_{32}Q_2^{1.85}}{d_{32}^{4.87}})
\end{equation}
Then using the equality constraints \ref{NewEC} and after some simplification, we can express the inequalities in \ref{Ineqc} in terms of $l_{11}$, $l_{21}$, and $l_{31}$. Notice that the equations are linear in the lengths, so we can express them in $Ax \leq B$ form.

\end{document}
